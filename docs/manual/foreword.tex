
\chapter*{Foreword}

This manual documents the release \version of Herodotos.
% It is organized as follows:
% \begin{itemize}
%   \item Part~\ref{part:usermanual} is an introduction to Herodotos
%   \item Part~\ref{part:refmanual} is the reference description
%     of Herodotos, its language and command line tool.
% \end{itemize}

%\section*{Conventions}

\section*{Copyright}

Herodotos copyright is\\
\copyright 2010, University of Copenhagen DIKU.\\
\copyright 2010, Nicolas Palix.\\

Herodotos is open source and can be freely redistributed under the
terms of the GNU General Public License version 2. See the file
\texttt{license.txt} in the distribution for licensing information.

The present documentation is copyright 2010, Nicolas Palix and
distributed under the terms of the GNU Free Documentation License
version 1.3.

\section*{Availability}

Herodotos can be freely downloaded
from \url{http://www.diku.dk/~npalix/herodotos/}.
This website contains also additional information.


%%% Local Variables:
%%% mode: LaTeX
%%% TeX-master: "herodotos.tex"
%%% coding: utf-8
%%% TeX-PDF-mode: t
%%% ispell-local-dictionary: "american"
%%% End:
