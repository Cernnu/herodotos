
\label{chap:howto}

At this point, you must have Herodotos installed. You now have to
setup your environment and write a Herodotos Configuration file.

\begin{enumerate}
\item Setup the environment
  \begin{itemize}
  \item Copy default make files and define your working directory. If
    you installed Herodotos from source, the default make files are
    \texttt{herodotos/scripts}. For a package install, copy the files
    from the Herodotos
    examples\footnote{\texttt{/usr/share/doc/herodotos/examples/scripts}}.
  \item Write a HC file which declares your projects, patterns and experiments.\\
    If you want to have several .hc files, you should define the CONF
    environment variable to select a default one. Every following make
    command will then use that configuration file.
  \end{itemize}

\item Recover missing elements in a version description, like a release
  date, the code size, or extract the code from a given repository:

  \texttt{make preinit}
  
\item Initialize the environment:

  \texttt{make init}
  
\item Compute raw data:

  \texttt{make -j2}

\item Correlate bug reports and propose an initial set of
  correlations:

  \texttt{make correl}

  Note: Herodotos will automatically compute the changeset.

\item Manually edit the remaining correlations

  \texttt{\$EDITOR results/*/*.correl.org}

  Emacs with the Org mode is recommended.
  Org mode view-link add-on will transform your
  emacs in a wizard to handle Coccinelle reports.

  See \texttt{./tools/org-view-link.el}

\item Use the new complementary correlation information

  \texttt{make correl} (recompute the correlation and use the
  correlation set manually provided)

\item Check the result:

  \texttt{make stat}

\item Look for false positive

  \begin{itemize}
  \item \texttt{cp results/*/*.edit.org results/*/*.new.org}
  \item Edit the new copy and annotate the reported TODO with either
    BUG or FP.
  \end{itemize}

\item Generate graphs (or website):

  \texttt{make [graph|web]}

\item Edit and adjust your HC file\\

  Iterate on steps 9 and 10.

\item Generate SQL queries from *.new.org files:
  
  \texttt{herodotos ---prefix prefix/ ---parse\_org results/*/*/*.new.org ---to-sql}
  
\item Generate incremental SQL queries from *.new.org files

  \texttt{herodotos ---prefix prefix/ ---parse\_org results/*/*/*.new.org ---to-sql-update version}
  
  Generated queries are used to update a data base containing data
  about all versions preceding the given version, with data about the
  given version and all versions following it.
  
  \textbf{Note}: For the last two commands, it's supposed that a
  database has already been created. To insert new data corresponding
  to generated queries, the output result of these commands can be
  redirected to a .sql file, which can be called once connected to the
  database. This process, that has to be done for each .new.org file,
  can be automated with a script that handles queries generation and
  the insertion of data corresponding to these queries.
\end{enumerate}


%%% Local Variables:
%%% mode: LaTeX
%%% TeX-master: "herodotos.tex"
%%% coding: utf-8
%%% TeX-PDF-mode: t
%%% ispell-local-dictionary: "american"
%%% End:
