\chapter{SQL modes}
\label{sec:sql-modes}

\section{SQL queries from a report}
\label{sec:sql-queries}
The to-sql mode of herodotos generates SQL queries from a report. Thus, the user has to run the to-sql mode of herodotos
with all reports, save the generated queries, and insert them in the database.
For example, the command
../herodotos --prefix /home/lotfi/RICM4/stage/dev/herodotos/herodotos/ --parse\_org results/testDemo/Test\_Error.new.org --to-sql
Has the following result:
\begin{lstlisting}
BEGIN;SET TRANSACTION ISOLATION LEVEL SERIALIZABLE;
INSERT INTO correlations (correlation\_id,  report\_error_name, status, reason\_phrase,
        data\_source)
	SELECT nextval('correlation\_idx'), 'Error', 'BUG', '', 'Test\_Error.new.org' FROM
        correlation\_idx;
INSERT INTO reports (correlation_id, file\_id, line\_no, column\_start, column_end,
        text\_link)
	SELECT last_value(correlation\_idx), get\_file('testDemo', 'v0/test.c'), 8, 1, 6,
	'I found an error' FROM correlation\_idx;
INSERT INTO reports (correlation\_id, file\_id, line\_no, column\_start, column_end,
        text\_link)
	SELECT last\_value(correlation\_idx), get\_file('testDemo', 'v1/test.c'),
	9, 1, 6, 'I found an error' FROM correlation\_idx;
INSERT INTO reports (correlation\_id, file\_id, line_no, column_start, column\_end,
        text\_link)
	SELECT last_value(correlation\_idx), get_file('testDemo', 'v2/test.c'), 9, 1, 6,
	'I found an error' FROM correlation\_idx;
INSERT INTO report\_annotations SELECT (SELECT report\_id FROM reports, correlation\_idx 
        WHERE file\_id=get\_file('testDemo', 'v0/test.c') AND
        correlation\_id=last\_value(correlation\_idx)
        AND line\_no=8 AND column\_start=1 AND column\_end=6), 8, 7, 10, 'err';
COMMIT;
\end{lstlisting}

This part of the result shows that each bug inserted in the database can be identified thanks to its correlation\_idx.
Consequently, a bug can be seen as a correlation, and what is insert into reports table is actually each version of
each bug.

\section{Incremental SQL queries from a report}
\label{sec:sql-queries}
When a study has been done on a large number of versions, and data have been inserted into the database, insert new data
corresponding to newer versions results with the to-sql mode involves the deletion of the database and the generation
of new queries from reports including the latest versions. To avoid that, SQL functions and some tests have been added
to allow updating a database since a given version. Thanks to bug identifiers, herodotos checks if the bug already
belongs to a correlaiton, if it does,it is inserted with this correlation identifier, 
else, it is inserted as a new correlation.
This mode is the to-sql-update mode.For the example we have, if a v3 and a v4  
versions are added to the study, the command to run to generate incremental queries is:
../herodotos --prefix /home/lotfi/RICM4/stage/dev/herodotos/herodotos/ --parse\_org results/testDemo/Test\_Error.new.org --to-sql-update v3.


Even if herodotos does not handle SQL connection and these modes need an execution of herodotos for each pattern,
the process can easily be automated thanks to scripts which run herodotos with one of these modes on each
report, redirect the output to a text file, and then give it as a database input.