
\chapter{Installation}
\label{sec:install}

\section{Requirements}
\label{sec:requirements}

\paragraph{To install Herodotos from the source}

You need the O'Caml compiler. Herodotos parsers are generated with
Menhir. However they have been pre-generated in the source package.

\paragraph{To use Herodotos}

You need the following:

\begin{itemize}
\item A org-mode capable matching tool. For instance, Coccinelle
  (spatch) is perfect.

\item Herodotos relies on EPS generation using jgraph and ps2eps. They
  are thus mandatory.
\item To convert to PDF files, Herodotos uses epstopdf which is part
  of texlive. You may also consider installing pdftk to manipulate
  PDF files.
\item To easy diffusion of your results on the web, Herodotos is able
  to generate a web site with images in PNG and SVG formats. If you
  want to use this feature, you need to have imagemagick (for PNG) and
  inkscape (for SVG). Note that the conversion to SVG needs the PDF
  version of your graphs. You thus need to have epstopdf
  available. Finally, this feature requires inscape version to be at
  least 0.47.
\item To improve the execution time of Coccinelle, you may consider
  using the glimpse indexer.
\end{itemize}

\section{Sources}
\label{sec:sources}

Sources are available on several web sites:
\begin{itemize}
\item \url{http://www.diku.dk/~npalix/herodotos/}
\item \url{https://launchpad.net/~npalix/+archive/herodotos}
\item \url{https://launchpad.net/~npalix/+archive/herodotos-stable}
\end{itemize}

\section{Distribution package}
\label{sec:distro-package}

If you're under Ubuntu, you may add \texttt{ppa:npalix/herodotos} or
\texttt{ppa:npalix/herodotos-stable} as an other software source in
Synaptic. In that case, runtime dependencies are automatically
handled.

For more information, you can go to the PPA pages on Launchpad
mentioned in previous section.

%%% Local Variables:
%%% mode: LaTeX
%%% TeX-master: "herodotos.tex"
%%% coding: utf-8
%%% TeX-PDF-mode: t
%%% ispell-local-dictionary: "american"
%%% End:
